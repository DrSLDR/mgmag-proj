\documentclass[a4paper]{llncs}
\usepackage{hyperref}
\usepackage[utf8]{inputenc}
%\usepackage{amsmath,amsthm,amssymb,MnSymbol}
%\usepackage{amsmath}
\usepackage{stmaryrd}
\usepackage{hyperref}
\usepackage{bussproofs}
\usepackage{booktabs}
\usepackage{color}
\usepackage{lipsum}
\usepackage{graphicx}

\begin{document}

\title{MGMAG Project}
\subtitle{Project Proposition}
%
%\titlerunning{}  % abbreviated title (for running head)
%                                     also used for the TOC unless
%                                     \toctitle is used
%
\author{Jonas A. Hultén\inst{1} \and Jappie Klooster\inst{2} \and Eva Linssen\inst{3} \and Deliang Wu\inst{4}}
%
%\authorrunning{} % abbreviated author list (for running head)
%
\institute{
  (5742722) \email{j.a.hulten@students.uu.nl}
\and
  (5771803) \email{j.t.klooster@students.uu.nl}
\and
  (3902749) \email{e.linssen@students.uu.nl}
\and
  (5622182) \email{d.wu@students.uu.nl}\\
Universiteit Utrecht, 3584 CS Utecht, The Netherlands}

\maketitle              % typeset the title of the contribution

\pagestyle{headings}

\section*{Project Goal}

We want to implement the board game ``Gravwell: Escape from the 9th Dimension'' in a simple graphical setting,
with the possibility of playing one human player versus AI players,
or all AI players against each other. The AI should be able to perform
better -- on long-run average -- than a player choosing random actions at
each turn.

We chose Gravwell as our game as, being an existing board game,
it allows us to focus more on the AI implementation,
since good gameplay is more-or-less already given. Furthermore,
we are not aware of any digital implementations of this game, so the
project will thus tread new ground. The game rules will be taken from the
official manual
\footnote{
	\url{http://www.cryptozoic.com/sites/default/files/icme/u2793/gvw_rulebook_final.pdf}
}.

\section*{Agent Qualities}

Fundamentally, the agent(s) must be able to play Gravwell.
In its most simplistic version, this will mean an agent which can select
and play a random card from its hand. Later, we might add an agent that chooses cards based on a set of rules or with the use of a decision tree. Our primary aim, however, is to create an agent which is capable of learning from its previous plays --
possibly across multiple games -- so that it can optimize its strategy. We can use the non-learning agents to evaluate the effectiveness of the learned strategy on.

The method of learning is not fixed; we are considering approaches ranging
from relatively simple reinforcement learning to evolutionary neural networks,
working either off- or online. If this is successful,
rudimentary agent-agent interaction such as taunting may be implemented.

Lastly, as Gravwell supports 2-4 players, our agent platform must support
1-4 agents, each with its own strategy and knowledge.

\section*{Planning Outline}
We'll use SCRUM as a project management tool.

\subsection*{Important dates}
The final deadline is July 1. Prior to that, a presentation of our project
must be ready by June 21 or 23. Additionally, since some team members are
leaving early, the final deadline for the project as a whole will be moved
back to June~27.

\subsection*{Planning of sprints}
\begin{tabular}{l l l}
	What & Date & Day \\ \toprule
	Finish startup sprint & 2016-05-10 & Tuesday\\
	Sprint planning Sprint II & 2016-05-11 & Wednesday\\
	Finish Sprint II, hand in result/present & 2016-05-24 & Tuesday\\
	Sprint planning Sprint III & 2016-05-25 & Wednesday\\
	Finish Sprint III, hand in result/present & 2016-06-7 & Tuesday\\
	Sprint planning Sprint IV& 2016-06-8 & Wednesday\\
	Finish Sprint IV, hand in result/present & 2016-06-21 & Tuesday\\ \bottomrule
\end{tabular} \\

\subsubsection*{First sprint}
The sprint planning for the first contains the GUI-development and
gathering of research material. Our strategy will be to gather as
many papers and other resources as we can in this sprint so we will
already have all the references we need.
The first sprint also contains creating the initial backlog.

As shown in the table above, our sprints will finish exactly on the first
date of final presentations. The end-of-sprint presentations may
be informal, directed at either just the teacher or the team.
The remaining time after the final presentations can be used to
resolve documentation.

\subsubsection*{Other sprints} The content of the other sprints will be
decided at their respective sprint plannings.
\end{document}
